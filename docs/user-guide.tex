% Daybo Logic Podcast downloader
% Copyright (c) 2012-2014, David Duncan Ross Palmer (M6KVM), Daybo Logic
% All rights reserved.
% 
% Redistribution and use in source and binary forms, with or without
% modification, are permitted provided that the following conditions are met:
% 
%     * Redistributions of source code must retain the above copyright notice,
%       this list of conditions and the following disclaimer.
% 
%     * Redistributions in binary form must reproduce the above copyright
%       notice, this list of conditions and the following disclaimer in the
%       documentation and/or other materials provided with the distribution.
% 
%     * Neither the name of the Daybo Logic nor the names of its contributors
%       may be used to endorse or promote products derived from this software
%       without specific prior written permission.
% 
% THIS SOFTWARE IS PROVIDED BY THE COPYRIGHT HOLDERS AND CONTRIBUTORS "AS IS"
% AND ANY EXPRESS OR IMPLIED WARRANTIES, INCLUDING, BUT NOT LIMITED TO, THE
% IMPLIED WARRANTIES OF MERCHANTABILITY AND FITNESS FOR A PARTICULAR PURPOSE
% ARE DISCLAIMED. IN NO EVENT SHALL THE COPYRIGHT OWNER OR CONTRIBUTORS BE
% LIABLE FOR ANY DIRECT, INDIRECT, INCIDENTAL, SPECIAL, EXEMPLARY, OR
% CONSEQUENTIAL DAMAGES (INCLUDING, BUT NOT LIMITED TO, PROCUREMENT OF
% SUBSTITUTE GOODS OR SERVICES; LOSS OF USE, DATA, OR PROFITS; OR BUSINESS
% INTERRUPTION) HOWEVER CAUSED AND ON ANY THEORY OF LIABILITY, WHETHER IN
% CONTRACT, STRICT LIABILITY, OR TORT (INCLUDING NEGLIGENCE OR OTHERWISE)
% ARISING IN ANY WAY OUT OF THE USE OF THIS SOFTWARE, EVEN IF ADVISED OF THE
% POSSIBILITY OF SUCH DAMAGE.

\documentclass{article}
\usepackage{etoolbox}
\apptocmd{\sloppy}{\hbadness 10000\relax}{}{}
\usepackage{hyperref}
\begin{document}
\title{dlpodget \input{.version}Users' Guide}\author{Daybo Logic}
\maketitle

% .ident is the changeset of the repository, generated by Makefile.
% It only exists if you have Mercurial installed.
% If you do not have a Mercurial checkout, or hg is not installed, it
% contains a username@hostname combination, so that we know that the
% documentation is not official.  This is primarily to ensure that we
% know which exact revision of docuentation is being referred to in
% support and/or community discussions.
\par\vspace*{\fill}
\begin{center}
\input{.ident}
\end{center}

\newpage
\par \textbf{\underline{Contents}}
\\
\href{#about}{1.~ About dlpodget}
\\
\href{#features}{1.0~ Feature Summary}
\\
\href{#contact}{1.1~ Contacting the author}
\\
\href{#daybologic}{1.2~ Who are Daybo Logic?}
\\
\href{#faq}{1.3~ FAQ}

\par \href{#listings}{2.0~ Configuration Options}\\

\newpage
\textbf{1.~ About dlpodget}
\\
The Daybo Logic Podcast downloader is designed to work as an unattended, regularly scheduled servant, which
will download all of your favourite RSS feeds in the background on your server.  Alternatively, it can be run
on-demand from your desktop computer.
\\
\textbf{1.0~ Feature Summary}
\begin{itemize}
\item User-configurable stream definitions
\item Persistent database and statistics mode
\item Object-orientated, extensible design, for programmers and power-users
\item Non-RSS support
\item Rate-limited downloads
\item Parallel download support as standard, with user-defined resource limits
\item Supported on GNU and Darwin systems
\item Archived feed support
\item Codec translation support
\item Partial download recovery
\item User-defined download order
\item User-configurable pausing between stream downloads
\item Tree validation mode
\end{itemize}
\textbf{1.1~ Contacting the author}
\\
The lead developer of dlpodget is David Duncan Ross Palmer, a C/C++
programmer since 1997, a BASIC programmer since 1989.  ~One who dabbleth with Perl and Java in his employment in latter years~
The author of the library can be contacted by visiting
the following web-site: \href{http://www.daybologic.co.uk/mailddrp/}{\url{http://www.daybologic.co.uk/mailddrp/}}.~
He especially welcome bug reports, feature requests and technical
questions.\\
\\
\textbf{1.2~ Who are Daybo Logic?}
\\
Daybo Logic started in 1997 as a software development company.~ It
then became a general troubleshooting company, as a sole trader, David
Duncan Ross Palmer and in 2006, it reverted to being purely a
programming hobby of mine.~ I still release all my software,
including source code on the Daybo Logic web-site at \href{http://www.daybologic.co.uk/}{www.daybologic.co.uk}.\\
and additionally, on the code-hosting service, operated by Atlassian
\href{https://bitbucket.org/daybologic}{www.bitbucket.org/daybologic}.\\
Atlassian in no way condone or are connected with this software.\\
\\
\textbf{1.3~ FAQ}
\\
Frequently Asked Questions
\\
\par Q - What is the difference between dlpodget and iTunes

A - dlpodget is primarily a backgrounded service, with low over-head, and is highly configurable,
rather than an application primariy concerned with user desktops.
\\
\par Q - Is there a GUI?

A - It presently has no graphical user interface, although we may bolt an optional control accessory
or easy config editor on later, if there is user-demand.
\\
\par Q - Is there a version for MacOS or Windows?

A - Mac OS X is supported, you may need to install some modules via CPAN
Windows is not supported.  I welcome any reports on trying to use dlpodget via ActiveState Perl,
or Redhat Cygwin, where it should theoretically work.
\\
\\
\textbf{2.0 Configuration options}
\\
The main section controls the configuration file generally, per-feed settings override it,
where applicable.  All other section names identify a particular feed, and override this section.
\begin{itemize} % config
\item main % main
\begin{itemize}
\item enable \verb=<1|0>=\newline default 1, set to 0 for a convenient way to ignore the entire file.
\\
\item noop \verb=<1|0>=\newline default 0, set to 1 to disable modification of local files.
\\
\item debug \verb=<1|0>=\newline default 0, Enable verbose maintainer information
\\
\item maxchildren \verb=<0-n>=\newline default 0, max concurrent downloads. nb. setting this to 1
is less efficient than 0, since a dedicated child is cloned for each feed.
-1 means no limit (no more than the number of feeds in the configuration).
\\
\item childdelay \verb=<0-n>=\newline Artificial pause between forking, to avoid performance penalty on launch
\\
\item retries \verb=<0-n>=\newline default 0, number of times to retry downloading a the stream or the feed.
\\
\item rsleep \verb=<0-n[r]>=\newline default 0, time in seconds to sleep after a failed read, add 'r' suffix
to randomise 1-n seconds, 0r is illegal.
\end{itemize} % main
\end{itemize} % config

\par For the lastest news check the~ \href{http://bitbucket.org/daybologic/dlpodget}{web
page}~ and update your documentation \& regularly,
remember
updates cost nothing.~ If you have an equiry of a more technical
nature
you can contact me directly at \href{http://www.daybologic.co.uk/mailddrp/}{\url{http://www.daybologic.co.uk/mailddrp/}}\href{mailto:palmer@overchat.org}{}


\par (C)Copyright 2012 -
\the\year
~David Duncan Ross Palmer, Daybo Logic. The library may only be used in accordance
with
the license, which is supplied in the COPYING file.\\


\end{document}
