% Daybo Logic Podcast downloader
% Copyright (c) 2012-2014, David Duncan Ross Palmer (M6KVM), Daybo Logic
% All rights reserved.
% 
% Redistribution and use in source and binary forms, with or without
% modification, are permitted provided that the following conditions are met:
% 
%     * Redistributions of source code must retain the above copyright notice,
%       this list of conditions and the following disclaimer.
% 
%     * Redistributions in binary form must reproduce the above copyright
%       notice, this list of conditions and the following disclaimer in the
%       documentation and/or other materials provided with the distribution.
% 
%     * Neither the name of the Daybo Logic nor the names of its contributors
%       may be used to endorse or promote products derived from this software
%       without specific prior written permission.
% 
% THIS SOFTWARE IS PROVIDED BY THE COPYRIGHT HOLDERS AND CONTRIBUTORS "AS IS"
% AND ANY EXPRESS OR IMPLIED WARRANTIES, INCLUDING, BUT NOT LIMITED TO, THE
% IMPLIED WARRANTIES OF MERCHANTABILITY AND FITNESS FOR A PARTICULAR PURPOSE
% ARE DISCLAIMED. IN NO EVENT SHALL THE COPYRIGHT OWNER OR CONTRIBUTORS BE
% LIABLE FOR ANY DIRECT, INDIRECT, INCIDENTAL, SPECIAL, EXEMPLARY, OR
% CONSEQUENTIAL DAMAGES (INCLUDING, BUT NOT LIMITED TO, PROCUREMENT OF
% SUBSTITUTE GOODS OR SERVICES; LOSS OF USE, DATA, OR PROFITS; OR BUSINESS
% INTERRUPTION) HOWEVER CAUSED AND ON ANY THEORY OF LIABILITY, WHETHER IN
% CONTRACT, STRICT LIABILITY, OR TORT (INCLUDING NEGLIGENCE OR OTHERWISE)
% ARISING IN ANY WAY OUT OF THE USE OF THIS SOFTWARE, EVEN IF ADVISED OF THE
% POSSIBILITY OF SUCH DAMAGE.

\documentclass{article}
\usepackage{etoolbox}
\apptocmd{\sloppy}{\hbadness 10000\relax}{}{}
\usepackage{hyperref}
\begin{document}
\title{dlpodget \input{.version}Developers' Guide}\author{Daybo Logic}
\maketitle

% .ident is the changeset of the repository, generated by Makefile.
% It only exists if you have Mercurial installed.
% If you do not have a Mercurial checkout, or hg is not installed, it
% contains a username@hostname combination, so that we know that the
% documentation is not official.  This is primarily to ensure that we
% know which exact revision of docuentation is being referred to in
% support and/or community discussions.
\par\vspace*{\fill}
\begin{center}
\input{.ident}
\end{center}

\newpage
\par \textbf{\underline{Contents}}
\\
\href{#intro}{1.~ Introduction}
\\
\href{#langs}{1.0~ Languages}
\\
\href{#struct}{1.1~ Code structure}
\\
\href{#struct}{1.2~ Working practices / team organisation}
\\

\newpage
\textbf{1.~ Introduction}
\\
Great that you're interested in contributing to the project!  However, please do read the user-guide
first, so that you aware of the meanings of all of the currently existing options available to users,
and the current aims of the project.
\\
\textbf{1.0~ Languages}
\begin{itemize}
\item Perl
\end{itemize}
TODO
\textbf{1.1~ Code structure}
TODO

\newpage
\textbf{1.2~ Working practices / team organisation}
\\
Before committing any code to the project, it is recommended that you create an account
on Atlassian Bitbucket, and fork the \href{https://bitbucket.org/daybologic/dlpodget}{master repository}
to your own account.  You should then only commit code to the \textbf{develop} branch.  You can then
submit pull requests to the master repository, and this acts a staging area and allows other developers
to code review and comment on the pull request before it is accepted.

You should never commit code to the \textbf{stable} or \textbf{default} branches.  These are special
branches, whose meaning is as follows:

\begin{itemize}
\item stable
\\
This branch is the freeze branch for properly unit tested and features we have agreed to release in
the next iteration of the project.  Code here is expected to be mature and mostly production ready.
Broken code in this branch shoud be avoided.  Users can test the 'bleeding edge' by using checkouts
from this branch, though they should not do so from \textbf{develop}
\item default
\\
The default branch contains the latest version, merged down from whichever the latest release tag is.
There must not be any commit within this branch except a merge commit.
\\
\end{itemize}
Please do not make named feature branches.  Always develop within \textbf{develop}, and use bookmarks
or additional repository clones for separate features.

\par For the lastest news check the~ \href{http://bitbucket.org/daybologic/dlpodget}{web
page}~ and update your documentation \& regularly.


\par (C)Copyright 2012 -
\the\year
~David Duncan Ross Palmer, Daybo Logic. The library may only be used in accordance
with
the license, which is supplied in the COPYING file.\\


\end{document}
